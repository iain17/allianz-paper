\chapter{Conclusie}\label{chap:conclusion}
Het type privé blockchain kan voor het proof of concept direct uitgesloten worden. Dit omdat het hele idee van het proof of concept gaat om het feit dat alle transacties transparant zijn voor alle deelnemers van het netwerk. Zodat bijvoorbeeld verzekeraars een transactie kunnen verifiëren voordat ze hem uitbetalen.\par

Hierna houden we alleen nog het Consortium en openbaar type blockchain over. Hierbij is gekozen voor consortium omdat deze efficiënter is in het verwerken van transacties. Daarnaast zit in de consortium variant het project Ethereum, die momenteel de enige is die smart contracts biedt op de blockchain.\par

Verder is de conclusie dat de technologie om de blockchain en hoe deze werken nu duidelijk en kan er ontwikkeld worden aan het proof of concept.