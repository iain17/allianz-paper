\chapter{Technische achtergrond}\label{chap:q1}
In dit hoofdstuk wordt er een korte uitleg gegeven over een aantal basis technische termen in cryptografie, blockchain-technologie en gerelateerde concepten zoals smart contracts.

\section{Blockchain-technologie}
De blockchain is een specifieke databasetechnologie die leidt tot een gedistribueerd autonoom grootboeksysteem (Kaptijn, B., Bergman, P., Gort, S. Whitepaper block-chain. ICTU, 2016). De integriteit van dit gedistribueerd autonoom grootboeksysteem wordt gewaarborgd doordat iedere partij zeggenschap heeft bij de validatie van een transactie. Dit versnelt het proces doordat beheerders en tussenpersonen worden uitgeschakeld. Meningsverschillen worden opgelost door een consensus van een meerderheid van de deelnemers.
\par
Databasetransacties worden gegroepeerd in blokken die vervolgens achter elkaar in een reeks blokken worden opgeslagen, vandaar de naam blockchain. De koppeling tussen blokken en hun inhoud wordt beschermd door cryptografie en kan niet worden vervalst. Daarom kan informatie die eenmaal in een blockchain is ingevoerd niet worden gewist; In essentie bevat een blockchain een accuraat, tijd gestempeld en verifieerbaar archief van elke transactie die ooit is gemaakt.
\par
De technologie lost verschillende problemen op die voorkomen bij het gebruik van traditionele gecentraliseerde database technologieën die in handen zijn van een instantie. Dit soort technologieën vereisen vertrouwen dat de beheerder zorgvuldig omgaat met de toegang of bewerkingen van de data. Verder dat de database toegankelijk is voor de belanghebbenden en dat de instantie er de volgende dag nog is. Deze problemen komen niet voor in een gedecentraliseerde blockchain database.

\section{Overeenstemming algoritmes}
Overeenstemming algoritmes zijn van het grootste belang voor blockchaintechnologie, omdat het doel van Bitcoin was om waarde over te dragen in een niet-gereguleerde, wantrouwende omgeving, waar een zekere manier om transacties te valideren nodig was. Het doel van het consensusalgoritme is ervoor te zorgen dat er één historie van transacties bestaat en dat die geschiedenis geen ongeldige of tegenstrijdige transacties bevat. Bijvoorbeeld dat geen account probeert meer uit te geven dan het account bevat, of om hetzelfde token twee keer uit te geven, de zogenaamde double-spending. In tabel 2.2 worden verschillende belangrijke consensusalgoritmen met elkaar vergeleken. Hieronder wordt een korte introductie gegeven van een paar van hen, maar voor meer details wordt de lezer verwezen (Back, 1997), (Nakamoto, 2008), (Fischer, 1983), (Tendermint, 2017).

\section{Smart contracts}
De naam slimme contracten is aantoonbaar een verkeerde benaming omdat ze in feite niet slim zijn noch contracten in gezond verstand. Slimme contracten zijn, in de context van blockchain, gewoon logica die op een blockchain wordt gepubliceerd, kan dergelijke transacties ontvangen of uitvoeren elk adres (transacties kunnen worden afgewezen of vereisen speciale argumenten om te functioneren) en dat kan fungeren als een onveranderlijke overeenkomst. Het doel van de slimme contracten is om op te treden als een "geautomatiseerd transactieprotocol dat de voorwaarden van een contract uitvoert" (Szabo, 1994) en werd voor het eerst bedacht door cryptograaf Nick Szabo. Het basisidee, en de bron van het contractdeel in de naam, is dat bepaalde delen van contracten kunnen zodanig in de software worden opgenomen dat de inbreuk daarop ofwel duur is of onmogelijk. Slimme contracten worden vaak verward met Ricardiaanse contracten (Griggs, 2015), de digitale opname en verbinding met andere systemen van een contract op wet. Dit is niet wat met slimme contracten wordt bedoeld, omdat ze niet legaal hoeven te zijn op geen enkele manier, noch verbonden met externe systemen. Men zou zich echter waarde kunnen voorstellen in de koppeling van slimme contracten met Ricardiaanse om de functionaliteit van "uit te besteden" juridische contracten met slimme contracten