\chapter{Inleiding}
Allereerst wordt in dit hoofdstuk het onderwerp van deze scriptie behandeld. Dit word gedaan door in paragraaf \ref{chap:motivation} de aanleiding van het onderzoek te bespreken. Waarna de relevantie in paragraaf \ref{chap:relevance} wordt besproken en aansluitend in paragraaf \ref{chap:researchQuestions} de doel- en vraagstellingen worden geformuleerd.

Het verdere verslag bestaat uit de resultaten van het onderzoek. Er is in het begin algemeen onderzoek gedaan naar de blockchain-technologie en gerelateerde onderwerpen zoals smart contracts. Waarna er daarna verslag wordt gedaan over de implementatie en architectuur van beide de blockchain technologie en het proof of concept dat is ontwikkeld. Waarmee in paragraaf \ref{chap:conclusion} de hoofdvraag wordt beantwoord. Hierna volgt een hoofdstuk voor verdere discussie en onderzoek.

\section{Aanleiding}\label{chap:motivation}
Zoals al aangegeven in het plan van aanpak \cite{pva} verzekeren verzekeringsmaatschappijen zoals Allianz panden voor miljoenen. Dit type verzekeringen worden gedeeld met meerdere verzekeraars, om zo het risico te verspreiden. Dit principe heet co-insurance en het probleem hiermee en ook gelijk de aanleiding voor dit onderzoek is dat het claimproces te veel tijd kost voordat deze wordt uitgekeerd naar de klant. Waardoor klanten van Allianz ontevreden zijn. Dit komt omdat dit proces door de verschillende instanties op verschillende handmatige manier worden uitgevoerd. Het proces wordt bijvoorbeeld bij Allianz gedaan met Excel bestanden, maar dit verschilt per verzekeringmaatschappij. Een claim kan dus vaak meer dan 3 maanden duren voordat deze werkelijk wordt uitbetaald.

\section{Relevantie}\label{chap:relevance}
De relevantie van dit onderzoek is om de laatste technologie op softwaregebied te onderzoeken en hiermee een proof of concept te ontwikkelen. In dit geval heeft de opdrachtgever aangegeven om in dit onderzoek naar de blockchain en smart contracts te willen kijken.

\newpage

\section{Probleemstelling}\label{chap:researchQuestions}
Het doel van deze scriptie is om aan te tonen hoe blockchaintechnologie en smart contracts gebruikt kunnen worden om informatie over claims van verzekeringen veilig te delen en te controleren tussen partijen die elkaar niet noodzakelijk vertrouwen.\par
Dit wordt bewezen door een proof-of-concept softwareapplicatie voor de use case van elektronische verzekeringsgegevens  te maken. Andere use-cases liggen buiten de scope van dit onderzoek.
\par
De hoofdvraag (\textbf{MRQ}) van dit onderzoek is:
\begin{center}
	\textbf{\researchQuestionMain}
\end{center}

Deze vraag is onderverdeeld in verschillende deelvragen (\textbf{SRQ}):
\begin{itemize}
	\item \textbf{SRQ1: \researchQuestionOne} om deze vraag te beantwoorden wordt er een  literatuuronderzoek uitgevoerd naar een technische basis termen rondom de blockchain-technologie en gerelateerde onderwerpen.
  \item \textbf{SRQ2: \researchQuestionTwo} om deze vraag te beantwoorden wordt er een software architectuur opgebouwd op basis van de wensen van de opdrachtgever, waaruit een literatuur onderzoek en vergelijking wordt uitgevoerd op verschillende oplossingsrichtingen en technologieën.
  \item \textbf{SRQ3: \researchQuestionThree} om deze vraag te beantwoorden wordt er een experiment uitgevoerd op basis van de kennis die is opgedaan in de vorige SRQ1 en SRQ2 deelvragen.
\end{itemize}

De conclusie van deze deelvragen beantwoorden de hoofdvraag van dit onderzoek in paragraaf \ref{chap:conclusion}. Het resultaat van het experiment dat in SRQ3 wordt uitgevoerd zal in dit verslag alleen bestaan uit de smart contracts code. De verdere backend en front-end code die is ontwikkeld wordt naast dit document meegeleverd maar is te groot om opgenomen te worden in de bijlagen.\par
\newpage