\chapter{Inleiding}
Dit onderzoeksverslag is geschreven tijdens het afstudeerproject van Iain Munro voor het Claim proces van Allianz. Allereerst wordt in dit hoofdstuk het onderwerp van dit onderzoeksverslag behandeld. Dit wordt gedaan door in paragraaf \ref{chap:motivation} de aanleiding van het onderzoek te bespreken. Waarna de relevantie in paragraaf \ref{chap:relevance} wordt besproken en aansluitend in paragraaf \ref{chap:researchQuestions} de doel- en vraagstellingen worden geformuleerd.\par

Het verdere verslag bestaat uit de resultaten van het onderzoek naar de blockchain-technologie en gerelateerde onderwerpen zoals smart contracts. Verder behandeld dit onderzoek ook de verschillende types blockchain. Hierna volgt de conclussie in paragraaf \ref{chap:conclusion}.

\section{Aanleiding}\label{chap:motivation}
De aanleiding voor dit onderzoek is dat er een proof of concept ontwikkeld wordt voor Allianz. Echter zijn de onderwerpen nog niet bekend. De aanleiding hiervoor is aangeven in het plan van aanpak \cite{pva}.

Zoals al aangegeven in het plan van aanpak \cite{pva} verzekeren verzekeringsmaatschappijen zoals Allianz panden voor miljoenen. Het zogeheten co-insurance wordt gedeeld met meerdere verzekeraars, om zo het risico te verspreiden. Het probleem met dit principe en ook gelijk de aanleiding voor dit onderzoek is dat het claimproces te veel tijd kost voordat deze wordt uitgekeerd naar de klant. Hierdoor daalt de klantentevredenheid van Allianz. Het proces moet namelijk op verschillende instanties op een handmatige manier worden uitgevoerd. Bij Allianz gedaan wordt het proces bijvoorbeeld uitgevoerd met Excel bestanden, maar dit verschilt per verzekeringsmaatschappij. Een claim kan dus vaak meer dan 3 maanden duren voordat deze werkelijk wordt uitbetaald.

\section{Relevantie}\label{chap:relevance}
De relevantie van dit onderzoek is om een aantal basis technische termen in cryptografie, blockchain-technologie en gerelateerde onderwerpen duidelijk te krijgen. Hiermee weet ik waar ik het over heb wanneer ik het proof of concept ontwikkel en hierover besluiten maak.\par

De relevantie van het onderzoek naar smart contracts is dat dit een term is die wordt gebruikt bij het ontwikkelen van een gedecentraliseerde blockchain applicatie.

\newpage

\section{Probleemstelling}\label{chap:researchQuestions}
Het doel van dit onderzoeksverslag is om aan te tonen hoe blockchain technologie en smart contracts gebruikt kunnen worden om informatie over claims van verzekeringen veilig te delen en controleren, vooral tussen partijen die elkaar niet noodzakelijk vertrouwen.\par
Dit wordt bewezen door een proof of concept softwareapplicatie voor de use case van elektronische verzekeringsgegevens te maken. Andere use-cases liggen buiten de omvang van dit onderzoek.
\par	
De hoofdvraag (\textbf{MRQ}) van dit onderzoek is:
\begin{center}
	\textbf{\researchQuestionMain}
\end{center}
Om deze vraag te beantwoorden wordt er een literatuuronderzoek uitgevoerd naar een technische basis termen rondom de blockchain-technologie en gerelateerde onderwerpen.
\newpage