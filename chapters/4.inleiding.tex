\chapter{Inleiding}
\section{Motivatie}
Verzekeringsmaatschappijen zoals Allianz verzekeren panden voor miljoenen. Dit type verzekeringen worden in de praktijk gedeeld met meerdere verzekeraars, om zo het risico te verspreiden. Dit principe heet co-insurance en de verzekeringen worden in een bestaand systeem genaamd E-ABS 1 opgeslagen door de verschillende verzekeraars.
Het probleem en ook gelijk de aanleiding voor dit onderzoek is dat het claimproces te veel tijd kost voordat deze wordt uitgekeerd naar de klant. Waardoor klanten van Allianz ontevreden zijn. Dit komt omdat het claimproces buiten E-ABS loopt waardoor een claim door de broker (makelaar) en de verschillende verzekeringsmaatschappijen met handmatige bedrijfsprocessen eerst gevalideerd moeten worden en gezamenlijk uitgekeerd. Het proces wordt momenteel bij Allianz gedaan met Excel bestanden, maar dit verschilt per verzekeringmaatschappij.
Een claim kan dus vaak meer dan 3 maanden duren voordat deze werkelijk wordt uitbetaald.

\section{Doel en onderzoeksvragen}
Het doel van dit scriptie is om aan te tonen hoe blockchaintechnologie en smart contracts gebruikt kunnen worden om informatie over claims van verzekeringen veilig te delen en te controleren tussen partijen die elkaar niet noodzakelijk vertrouwen.\par
Dit wordt bewezen door een proof-of-concept software applicatie voor de use case van elektronische verzekeringgegevens. De resultaten van het onderzoek kan buiten de scope van dit onderzoek toegepast worden voor andere usecases. Het doel van het onderzoek kan worden opgesplitst in de volgende drie onderzoeksvragen:
\begin{itemize}
  \item \textbf{Onderzoeksvraag 1: Uit welke use cases, requirements en concerns bestaat het huidige proces van Allianz?}
Om deze vraag te beantwoorden, zal ook een literatuuronderzoek worden uitgevoerd. Verder zal er een onderzoek worden gedaan naar de bestaande frameworks en technologieën voor data-opslag (met en zonder de blockchain technologie). Het laatste deel zal worden gedaan door een vergelijking uit te voeren op de verschillende technologieën.

  \item \textbf{Onderzoeksvraag 2: Welke kansen \& knelpunten bestaan bij het toepassen van de blockchain?} -
  \item \textbf{Onderzoeksvraag 3: Hoe automatiseer je het huidige proces met de blockchain?} -
\end{itemize}
\newpage

\section{Limitaties}
This PoC will strictly consist of the code necessary for the smart contracts, which
define most of the operational logic and basic permissions management. Considering
the time constraints of this thesis (approximately six months) and abundance of
existing blockchains, no blockchain will be programmed. However, in Section 3.2.4
and in 2.2.3, there is a discussion of blockchains design considerations for the extension
of the PoC. The thesis discusses cryptography used in blockchains and some
additional encryption mechanisms are suggested for the PoC. These are however
relying on existing technologies and implementations and are not part of the smart
contracts code. One could also argue that other parties involved in the economics
and regulation of health care such as insurance companies, the Ministry of public
health or the medical products agency should be included. Although these types
of users are not implemented with their specific requirements, in the PoC, they are considered and discussed in Section 5.2. Another highly relevant subject, important
for the application of block-chain technology to handling of personal data such as in
the medication plan, are legal considerations. Since this is a technical thesis, most
legal requirements are not discussed but the ambition is that data privacy laws shall
be honoured.

\section{Gerelateerd werk}
Since the start of this thesis (August 2016), much related work has been done,
advances in blockchain technology and large open source efforts in development have been made. (Zyskind et al., 2015), presents ENIGMA, a blockchain-based solution for secure multi-party computations. They suggest using blockchains for permissions management and for storing pointers to encrypted data, while the actual data is hosted by a trusted, blind escrow service. (Kosba, Miller, Shi, Wen, & Papamanthou, 2015) lay the groundwork for a project called HAWK, a framework and compiler for writing privacy-preserving smart contracts. (Kish & Topol, 2015) Propose in Nature Biotechnology, the use of blockchain technology for managing patient data but do not discuss a specific implementation or technical discussion. (Azaria, Ekblaw, Vieira, & Lippman, 2016) design a modular system for storing electronic medical records on a blockchain, they suggest a Proof-of-Work system for incentivising the participation of doctors and hospitals in the system. Med-Vault (“Medical Records Project Wins Top Prize at Blockchain Hackathon,” 2015) were present in the media but have not published any details regarding their blockchain-EMR. To the best knowledge of the author, there have been no functional, electronic medical prescriptions based on blockchain technology built so far