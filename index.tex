%%%%%
%%
%% Onderzoeksverslag
%%
%% Version: v0.1
%% Authors: Iain Munro
%% Date: 18/02/2018
%%%%%

%Install packages using:
%sudo tlmgr install

% Available documentclass options:
%
%   <all `report` document class options, e.g.: `a5paper`>
%   withindex   - enables the index. New index entries can be added through `\index{my entry}`
%   glossary    - enables the glossary.
%   techreport  - typesets the thesis in the technical report format.
%   firstyr     - formats the document as a first-year report.
%   times       - uses the `Times` font.
%   backrefs    - add back references in the Bibliography section
%
% For more info see `README.md`
\documentclass[firstyr,a4paper,oneside]{cam-thesis}%withindex
\usepackage[dutch]{babel}
% Citations using numbers
\usepackage[numbers]{natbib}

\newcommand{\thesisTitle}{Allianz - Automatiseren Claim Process}

\usepackage[utf8]{inputenc}


\newcommand{\researchQuestionMain}{``\textit{Hoe is de blockchain technologie in te zetten om het claimproces van Allianz te automatiseren?}''}

\newcommand{\researchQuestionOne}{``\textit{Wat is de blockchain en hoe werkt het?}''}
\newcommand{\researchQuestionTwo}{``\textit{Uit welke use cases, requirements en concerns bestaat het te ontwikkelen proof of concept?}''}
\newcommand{\researchQuestionThree}{``\textit{Welke kansen & knelpunten bestaan bij het toepassen van de blockchain?}''}

%APA norm
\usepackage{babel} 
\usepackage{apalike}
\bibliographystyle{apalike}

%tables new lines
\usepackage{makecell}

%Set the title spacing correctly
\usepackage{titlesec}
\titlespacing{\chapter}{0pt}{0pt}{0pt}
\titlespacing{\section}{0pt}{0pt}{0pt}
\titlespacing{\subsection}{0pt}{0pt}{0pt}

\input{solidity-highlighting.tex}

%Om de pagina margins etc te debuggen
%\usepackage{showframe}

\usepackage{pgfgantt}
\usepackage{rotating}
\usepackage[graphicx]{realboxes}

%%%%%%%%%%%%%%%%%%%%%%%%%%%%%%%%%%%%%%%%%%%%%%%%%%%%%%%%%%%%%%%%%%%%%%%%%%%%%%%%
%% Style (Changing the visual style of chapter headings and stuff.)
%%
\RequirePackage{titlesec}
% [Fixes issue #34 (see https://github.com/cambridge/thesis/issues/34). Solution from: http://tex.stackexchange.com/questions/299969/titlesec-loss-of-section-numbering-with-the-new-update-2016-03-15
\RequirePackage{etoolbox}
\makeatletter
\patchcmd{\ttlh@hang}{\parindent\z@}{\parindent\z@\leavevmode}{}{}
\patchcmd{\ttlh@hang}{\noindent}{}{}{}
\makeatother
% end of issue #34 fix]
\newcommand{\PreContentTitleFormat}{\titleformat{\chapter}[display]{\scshape\Large}
{\Large\filleft\MakeUppercase{\chaptertitlename} \Huge\thechapter}
{1ex}
{}
[\vspace{1ex}\titlerule]}
\newcommand{\ContentTitleFormat}{\titleformat{\chapter}[display]{\scshape\huge}
{\Large\filleft\MakeUppercase{\chaptertitlename} \Huge\thechapter}
{1ex}
{\titlerule\vspace{1ex}\filright}
[\vspace{1ex}\titlerule]}
\newcommand{\PostContentTitleFormat}{\PreContentTitleFormat}
\PreContentTitleFormat

%Om dubbelen legen pagina's weg te halen.
\let\cleardoublepage=\clearpage

%%%%%%%%%%%%%%%%%%%%%%%%%%%%%%%%%%%%%%%%%%%%%%%%%%%%%%%%%%%%%%%%%%%%%%%%%%%%%%%%
%% Thesis meta-information
%%

%% The title of the thesis:
\title{Automatiseren Claim Process}

%% The full name of the author (e.g.: James Smith):
\author{Calum Iain Munro}

%% College affiliation:
\college{Software Development, ICA, VT}

%% College shield [optional]:
\collegeshield{CollegeShields/ICA}

%% Submission date [optional]:
\submissiondate{July, 2018}

%% You can redefine the submission notice [optional]:
\submissionnotice{
HBO bachelorscriptie:\\
\textbf{Versie: 1 (Draft)}\\
\textbf{Datum: \today}\\
\\~\\
\textbf{Gegevens opdrachtgever:}\\
Bedrijf:			HeadForward B.V.\\
Contactpersonen:	Dani\"el Siahaya\\
\\~\\
\textbf{Gegevens opleiding:}\\
Opleiding: HBO bachelor Informatica\\
School: Hogeschool van Arnhem en Nijmegen\\
Begeleider:	Misja Nabben\\
Assessor: Rein Harle\\
\\~\\
\textbf{Gegevens opdrachtnemer:}\\
Teamlid: Calum Iain Munro (549288)\\
}

%% Declaration date:
\date{Febuari, 2018}

%% PDF meta-info:
\subjectline{Blockchaion and smart contracts}%Computer Science
\keywords{Onderzoeksverslag scriptie Calum Iain Munro HAN}

% %%%%%%%%%%%%%%%%%%%%%%%%%%%%%%%%%%%%%%%%%%%%%%%%%%%%%%%%%%%%%%%%%%%%%%%%%%%%%%%%
% %% Abstract:
% %%

 \abstract{My abstract ...}

% %%%%%%%%%%%%%%%%%%%%%%%%%%%%%%%%%%%%%%%%%%%%%%%%%%%%%%%%%%%%%%%%%%%%%%%%%%%%%%%%
% %% Acknowledgements:
% %%
 \acknowledgements{My acknowledgements ...}

%%%%%%%%%%%%%%%%%%%%%%%%%%%%%%%%%%%%%%%%%%%%%%%%%%%%%%%%%%%%%%%%%%%%%%%%%%%%%%%%
%% Glossary [optional]:
%%
% \newglossaryentry{HOL}{
%     name=HOL,
%     description={Higher-order logic}
% }

%%%%%%%%%%%%%%%%%%%%%%%%%%%%%%%%%%%%%%%%%%%%%%%%%%%%%%%%%%%%%%%%%%%%%%%%%%%%%%%%
%% Inhoudsopgave:
%%
\begin{document}
%%%%%%%%%%%%%%%%%%%%%%%%%%%%%%%%%%%%%%%%%%%%%%%%%%%%%%%%%%%%%%%%%%%%%%%%%%%%%%%%
%% Title page, abstract, declaration etc.:
%% -    the title page (is automatically omitted in the technical report mode).
\frontmatter{}

%Normale paragraven
\setlength{\parindent}{0em}
\setlength{\parskip}{1em}

%\chapter{Versiebeheer}
%\small
%\begin{center}
% \begin{tabular}{|c c c c|} 
% \hline
% Datum & Versie & Door wie & Aanpassing \\ [0.5ex] 
% \hline
% 12-03-2018 & v0  & Iain Munro & Eerste opzet \\
% \hline
%\end{tabular}
%\end{center}
%\end{small}
%\input{chapters/2.Voorwoord}
%\input{chapters/3.Samenvatting}
\chapter{Inleiding}
Dit onderzoeksverslag is geschreven tijdens het afstudeerproject van Iain Munro voor het Claim proces van Allianz. Allereerst wordt in dit hoofdstuk het onderwerp van dit onderzoeksverslag behandeld. Dit wordt gedaan door in paragraaf \ref{chap:motivation} de aanleiding van het onderzoek te bespreken. Waarna de relevantie in paragraaf \ref{chap:relevance} wordt besproken en aansluitend in paragraaf \ref{chap:researchQuestions} de doel- en vraagstellingen worden geformuleerd.\par

Het verdere verslag bestaat uit de resultaten van het onderzoek naar de blockchain-technologie en gerelateerde onderwerpen zoals smart contracts. Verder behandeld dit onderzoek ook de verschillende types blockchain. Hierna volgt de conclussie in paragraaf \ref{chap:conclusion}.

\section{Aanleiding}\label{chap:motivation}
De aanleiding voor dit onderzoek is dat er een proof of concept ontwikkeld wordt voor Allianz. Echter zijn de onderwerpen nog niet bekend. De aanleiding hiervoor is aangeven in het plan van aanpak \cite{pva}.

Zoals al aangegeven in het plan van aanpak \cite{pva} verzekeren verzekeringsmaatschappijen zoals Allianz panden voor miljoenen. Het zogeheten co-insurance wordt gedeeld met meerdere verzekeraars, om zo het risico te verspreiden. Het probleem met dit principe en ook gelijk de aanleiding voor dit onderzoek is dat het claimproces te veel tijd kost voordat deze wordt uitgekeerd naar de klant. Hierdoor daalt de klantentevredenheid van Allianz. Het proces moet namelijk op verschillende instanties op een handmatige manier worden uitgevoerd. Bij Allianz gedaan wordt het proces bijvoorbeeld uitgevoerd met Excel bestanden, maar dit verschilt per verzekeringsmaatschappij. Een claim kan dus vaak meer dan 3 maanden duren voordat deze werkelijk wordt uitbetaald.

\section{Relevantie}\label{chap:relevance}
De relevantie van dit onderzoek is om een aantal basis technische termen in cryptografie, blockchain-technologie en gerelateerde onderwerpen duidelijk te krijgen. Hiermee weet ik waar ik het over heb wanneer ik het proof of concept ontwikkel en hierover besluiten maak.\par

De relevantie van het onderzoek naar smart contracts is dat dit een term is die wordt gebruikt bij het ontwikkelen van een gedecentraliseerde blockchain applicatie.

\newpage

\section{Probleemstelling}\label{chap:researchQuestions}
Het doel van dit onderzoeksverslag is om aan te tonen hoe blockchain technologie en smart contracts gebruikt kunnen worden om informatie over claims van verzekeringen veilig te delen en controleren, vooral tussen partijen die elkaar niet noodzakelijk vertrouwen.\par
Dit wordt bewezen door een proof of concept softwareapplicatie voor de use case van elektronische verzekeringsgegevens te maken. Andere use-cases liggen buiten de omvang van dit onderzoek.
\par	
De hoofdvraag (\textbf{MRQ}) van dit onderzoek is:
\begin{center}
	\textbf{\researchQuestionMain}
\end{center}
Om deze vraag te beantwoorden wordt er een literatuuronderzoek uitgevoerd naar een technische basis termen rondom de blockchain-technologie en gerelateerde onderwerpen.
\newpage
\chapter{De blockchain technologie}\label{chap:q1}
In dit hoofdstuk wordt er een korte uitleg gegeven over een aantal basis technische termen in cryptografie, blockchain-technologie en gerelateerde concepten zoals smart contracts. Dit zodat we weten waar we het over hebben wanneer we het proof of concept behandelen.

\section{Het algemene concept achter de blockchain}
Blockchain technlogie is bekend geworden door de introductie van de digitale valuta bitcoin. De bitcoin werd geintroduceerd door Satoshi Nakamoto in 2008 door een white paper genaamed Bitcoin: A Peer-To-Peer Electronic Cash System \cite{bitcoinPaper}. Hierin legt hij uit hoe in een peer-to-peer omgeving geld overgemaakt kan worden om online betalingen rechtstreeks van de ene partij naar de andere overgemaakt kunnen worden, zonder een financiële instelling. De blockchain is de technologie achter de Bitcoin die het mogelijk maakt.\par

In dit onderzoek gebruiken we de beschrijving het ICTU \footnote{https://www.ictu.nl/}, die de blockchain beschrijft als: een specifieke databasetechnologie die leidt tot een gedistribueerd autonoom grootboeksysteem \cite{kaptijn}. De integriteit van dit gedistribueerd autonoom grootboeksysteem wordt gewaarborgd doordat iedere partij zeggenschap heeft bij de validatie van een transactie. Dit versnelt het proces doordat beheerders en tussenpersonen worden uitgeschakeld. Meningsverschillen worden opgelost door een consensus van een meerderheid van de deelnemers.\par

Databasetransacties worden gegroepeerd in data blokken die vervolgens achter elkaar in een reeks blokken worden opgeslagen, vandaar de naam blockchain. De koppeling tussen blokken en hun inhoud wordt beschermd door cryptografie en kan niet worden vervalst. Daarom kan informatie die eenmaal in een blockchain is ingevoerd niet worden gewist; In essentie bevat een blockchain een accuraat, tijd gestempeld en verifieerbaar archief van elke transactie die ooit is gemaakt. Figuur \ref{fig:blockchain?} geeft het algemene idee weer van hoe deze technologie werkt met als bekende use case de bitcoin.
\begin{figure}
    \begin{center}
        \includegraphics[scale=0.80]{images/blockchain?}
        \caption{Illustreert hoe de blockchain werkt \cite{howBlockchainWorks}}
        \label{fig:blockchain?}
    \end{center}
\end{figure}
\newpage

De blockchain technologie lost verschillende problemen op die voorkomen bij het gebruik van traditionele gecentraliseerde database technologieën die in handen zijn van één instantie. Dit soort technologieën vereisen vertrouwen dat de beheerder zorgvuldig omgaat met de toegang of bewerkingen van de data. Verder dat de database toegankelijk is voor de belanghebbenden en dat de instantie er de volgende dag nog is. Deze problemen komen niet voor in een gedecentraliseerde blockchain database. Dit komt omdat een nieuwe dienst, software bedrijf of markten op de blockchain de volgende zes design principes \cite{blockRev} hanteren:
\begin{enumerate}
	\item Netwerk integriteit\\
	Het systeem bewaakt de data integriteit doordat ieder lid in het netwerk alle transacties kan nalopen en kan controleren. Inplaats dat er maar een lid is dit proces uitvoert. Gebruikers op het netwerk kunnen rechtstreeks waarde met elkaar uitwisselen door dit te registeren op een blok. Elk blok heeft een verwijzing naar een voorgaand blok verwijzen, waardoor niemand een transactie kan verbergen of kan vervalsen. Dit omdat er meer andere gebruikers zijn met de juiste realiteit.
	\item Gedistribueerd\\
	Het systeem is volledig Gedistribueerd. Dit houd in dat er geen één punt is van controle. Er is niet een gebruiker of organisatie die het systeem uit kan zetten.
	\item Security\\
	In Satoshi’s white paper \cite{bitcoinPaper}, geeft hij aan dat iedere deelnemen van het netwerk vereist zijn om een public key infrastructure (PKI) te gebruiken om het platform veilig te houden. De PKI is een geavanceerde vorm van asymmetrische cryptografie, waar de gebruiker beide een publiek en privé sleutel ontvangt om zichzelf binnen het netwerk te identificeren en berichten kan versleutelen en onsleutelen.
	\item Eigendomsrechten\\
	Eigendomsrechten zijn transparant en afdwingbaar voor iedere gebruiker. Hierdoor dient een blockchain als een publiek register. Door een tool die Proof of Existence (PoE) heet. Deze tool creert en registeert de cryptografische overzichten van akten, licenties en andere rechten van gebruikers. Dit word gedaan door een hash te berekenen van de public key van een gebruiker.
	\item Privacy\\
	Gebruikers beheren hun eigen data. Er is geen centrale partij die dit doet. Op een blockchain netwerk kunnen gebruikers er zelf voor kiezen wat zij vrij geven aan persoonlijke informatie. Dit kan worden gedaan door persoonlijke gegeven mee te geven in hun public key of in een externe centrale database. Het gehele identificatie en verificatie laag die ervoor weggeeft wie wat naar elkaar stuurt is los van het de transactie laag. Hierdoor kunnen gebruikers op de blockchain anoniem zijn.	
	\item Valuta als motivatie\\
	Het systeem motiveert deelnemers van het netwerk door ze valuta te geven voor bepaalde acties op het netwerk. In het geval van de bitcoin, krijgen miners bitcoin geld voor het eerst volgende blok te koppelen aan het vorige blok aan data. Dit wordt gedaan door een cryptografische puzzel op te lossen die geleidelijk lastiger word.
\end{enumerate}
\newpage

\section{Type Blockchain}
Om een geïnformeerd besluit te maken welk type blockchain juist is voor het proof of concept worden de verschillende type blockchain in dit paragraaf behandeld. Dit word gedaan vanaf een hoog technisch niveau. In essentie zijn er drie types blockchain: privé, consortium en openbaar. Deze types kunnen daarna weer onderverdeeld worden in twee categorieën: open bevat openbaar en gesloten bevat privé en consortium. \par

De categorie gesloten, waarin de types privé en consortium zitten zijn bedoelt voor een gelimiteerde omgeving zoals een bedrijf of groepen bedrijven en organisaties. Terwijl een openbare blockchain een open is tot iedereen er geen permissies zijn die mensen of systemen erbuiten houden.\par

Per type blockchain word er ook gelijk gekeken naar de grote projecten die relevant zijn. Zodat in een vervolg hoofdstuk hierover een vergelijking kan worden uitgevoerd.

\subsection{Privé Blockchain}
Voor een volledige private blockchain, moeten de schrijf rechten op een centrale plek staan die beheert word vaak door een organisatie. De lees rechten kunnen beide publiekelijk of ook net zo beperkt zijn als de schrijf rechten. Applicaties die gebruik maken van een privé Blockchain zijn interne apps die alleen gebruikt worden binnen een bedrijf of organisatie. Want in andere gevallen word publieke lees rechten en controleerbaarheid vereist \cite{privateBlockChains}. Voorbeelden van privé Blockchains zijn MultiChain en Hyperledger:

\textbf{MultiChain} - MultiChain is een platform voor het ontwikkelen en publiceren van privé blockchains. Het lost een aantal schaalbaarheid problemen \cite{oreillyScalability} van de blockchain op met een geinteregeerde gebruiker permissie systeem. Verder bied het bedrijven de mogelijkheid om zonder software ontwikkelaars een blockchain op te richten \cite{mutlichain}.

\textbf{Hyperledger} - Hyperledger is een open source project die ontwikkeld word met het doel om een geavanceerd bedrijfstakoverkoepelende blockchain implementatie te ontwikkelen. Het word gehost door de Linux Foundation \cite{linuxFoundation} en wordt gezamelijk ontwikkeld door grote organisatie in financiën, banken, IoT, productie en technologie \cite{hyperledger}. 

\subsection{Consortium Blockchain}
Consortium blockchain is gedeeltelijk Privé. Het Overeenstemmingproces ookwel consensusproces over de integriteit van de data op de blockcahin wordt door een aantal voorafgeselecteerde nodes (gebruikers) uitgevoerd. Deze nodes zijn bijvoorbeeld 10 grote financiële instellingen die bij de aanmaak van een nieuwe block aan de blockchain zeggenschap hebben. Andere deelnemers van de blockchain hebben nogsteeds het recht om de besluiten van de 10 nodes te controlleren, maar ze hebben verder geen stem in het feit of de volgende block valide is.\par

Het voordeel van de consortium variant is deze efficiënter zijn en toch voldoende transactie transparantie geven. Ook is het niet een bedrijf die alleen oordeelt over de data. Voorbeelden van dit type zijn Ethereum en R3:

\textbf{Ethereum} - Het Ethereum project beschrijft zich als een gedecentraliseerd platform voor applicaties die precies gedraait worden zoals ze geprogrammeerd worden zonder enige kans van fraude, censuur of veranderingen van derden \footnote{https://www.ethereum.org/}. Applicaties, smart contracts, worden geprogrammeerd in de Solidity taal die voor het Ethereum project is ontwikkeld. Het word open-source ontwikkeld en er is een bondgenootschap, de Enterprise Ethereum Alliance \footnote{https://entethalliance.org/} dat bestaat uit 315 furtune 500 bedrijven en organisaties, die gezamelijk werken aan het het enige platform die smart contracts ondersteund op de blockchain.\cite{ethWood}\par

Het project kan gezien worden als verder uitgewerkte versie van Bitcoin die meer functionaliteiten toevoegt. Zo bestaat de status van Ethereum netwerk net zoals de Bitcoin uit meerdere objecten die 'accounts' worden genoemd, waarbij elke account een adres van 20 bytes en statusovergangen heeft. De staat van deze objecten worden opgeslagen in de blockchain waaruit gelijk afgeleiden kan worden waar valuta naartoe gaat.\cite{whitePaperEthereum}

\textbf{R3} - Dit is een gedistribueerd database-technologiebedrijf in New York. Het is verbonden met veel van 's werelds grootste financiële instellingen, met als missie om de voordelen van de blockchain te realiseren \cite{R3}.

\subsection{Openbare Blockchain}
Dit type blockchain is zoals de naam al aangeeft publiek beschikbaar tot iedereen in de wereld. Dit houd in tegenstelling tot Consortium ook het consensusproces in. Iedere gebruiker op het netwerk heeft zeggenschap op de geldigheid van nieuwe data en kan nieuwe data schrijven. \par

Een volledig publieke blockchain is een open-source systeem door het gebruik van zogeheete cryptoeconomics gebruikers op economisch doeleinde motiveert om samen te werken. Hierdoor kunnen ontwikkelaars die gebruik maken van zo'n blockchain belangen zoals beschikbaarheid waarborgen. Bijvoorbeeld zorgt een hogere tarief in een transactie resulteren in snellere transacties of convergentie over de nieuwe blokken die toegevoegd worden aan de blockchain. Een voorbeeld van hiervan is het bekende Bitcoin project.

\textbf{Bitcoin}. Bitcoin is het bekenste voorbeeld van een blockchain project. Bitcoin staat vooral bekend als digitale valuta en online betalingssysteem die door gebruik van cryptografie om valuta-eenheden te geneert en reguleert. Verder gebruikt het cryptografie om de overdracht van fondsen te verifiëren zonder een centrale bank. \par

\newpage
\chapter{De architectuur van de blockchain technologie}
Dit hoofdstuk gaat iets verder in op de basis architectuur van de blockchain. Dit zodat tijdens het ontwikkelen van de Proof of Concept we de basis begrippen van de architectuur begrijpen.

\section{Blok (block)}
De blockchain bied een gedistribueerd grootboeksysteem. Gegevens worden permanent opgeslagen in het netwerk via bestanden die blokken worden genoemd. Een blok is een document alle recente transacties die nog moeten worden vastgelegd. Het heeft daarom de naam blockchain, omdat het een reeks van blokken die naar de vorige verwijzen\cite{blochchainTechSymmbioticDev}.\par

Een blok in het geval van de Bitcoin bestaat uit een header en een body \cite{blockchainIssuesAndChallenges}. De header bestaat uit drie stukken meta gegevens. De eerste is een verwijzing naar een vorige blokhash (Merkle-hash\footenote{https://en.wikipedia.org/wiki/Merkle_tree}). Hierdoor verbindt het blok met de vorige uit de blockchain. De tweede set van meta gegevens is  moelijkheidsgraad, tijdstempel en nonce. Het laatste stuk metadata is de Merkle-tree root, een datastructuur die wordt gebruikt om alle transacties in het blok efficiënt samen te vatten \cite{masteringBitcoin}.

\section{Gedecentraliseerd netwerk}
The interactions among user on blockchain principally use a decentralized network in which each user represents a node at which a blockchain client is installed. When a user performing a transaction with another user or when a node receives data from another node, it verifies the authenticity of the data. It then broadcasts the validated data to every other node connected to it [86]. Within such a mechanism, the data spreads across the whole network. The benefit of using this mechanism is the centralization of the human factor is minimized and trust shifts from the human agents of a central organization to an open source code [5].

\section{Consensus (overeenstemming) algoritmes}
Om de werking van de blockchain te begrijpen en te vertrouwen, moet het begrip van Consensus oftewel overeenstemmings algoritmes duidelijk zijn. Deze algoritmes worden gebruikt wanneer een (nieuw) blok aan informatie geverifieerd wordt. Het zorgt voor één historie van transacties waar de geschiedenis geen ongeldige of tegenstrijdige transacties bevat.\par

Dit is allemaal nodig omdat de blockchain draait in een zelf gereguleerde, wantrouwende omgeving waar het nodig is om meningsverschillen over transacties binnen het netwerk op een lijn te krijgen. Het zorgt er bijvoorbeeld ook voor dat er niet één account is die meer uitgeeft dan dat het heeft, of waar hij of zij twee keer iets overmaakt, dit heet double-spending. De bekende consensus algoritmes zijn proof of work en proof of stake.\par

\begin{enumerate}
	\item Proof of Work (PoW)\\
	Het PoW consensus algoritme is het meest voorkomende algoritme in blockchain. Het werd geïntroduceerd door de Bitcoin en gaat ervan uit dat alle peers met rekenkracht mee stemmen door PoW-instanties, crytografische puzzels op te lossen en hiermee het recht hebben om de volgende blok aan te maken in het netwerk. Zo maakt de Bitcoin gebruik van een hash-gebaseerde PoW, wat inhoudt dat de peers een nonce-waarde \footnote{https://en.wikipedia.org/wiki/Cryptographic_nonce} proberen te vinden. Hierbij is wel de voorwaarden dat de vorige blokhash kleiner moet zijn dan de huidige doelwaarde die in de blokparameters staat van het vorige blok. Wanneer een dergelijke nonce wordt gevonden, maakt de miner het blok aan en stuurt hij het door naar zijn peers. Deze peers ontvangen dit dan en verifiëren of het klopt aan de hand van het vorige blok \cite{securityPOW}.
	\item Proof-of-Stake (PoS)\\
	Op het moment moet Proof-of-Stake zich nog bewijzen in de crypto valuta gemeenschap. Het is ontwikkeld om bestaande inefficiënte consensus algoritmes zoals PoW te vervangen. Het algemeen begrip van PoS is dat een peer (deelnemer van de blockchain), pas het stemrecht heeft op een nieuwe blok in de blockchain als de peer voldoende heeft ingezet in het netwerk. In het geval van PeerCoin \footnote{https://peercoin.net/} worden nieuwe blokken gegeneerd door het netwerk op basis van niet gespendeerde valuta en hoe oud deze is \cite{posProtocol}.\par
	
	Met deze methode wordt aangenomen dat mensen met meer valuta minder snel het netwerk zullen aanvallen \cite{blockchainIssuesAndChallenges}. Dit lost op het gebied van energiebesparing de problemen van PoW op, waar gebruikers miners aan zetten om valuta te ontvangen. Bij PoS wordt de valuta die niet beweegt steeds meer waard.
\end{enumerate}

\section{Smart contract}
De naam smart contract (slimme contracten) is aantoonbaar een verkeerde benaming omdat ze in feite niet slim zijn noch contracten in gezond verstand. Slimme contracten zijn, in de context van blockchain, gewoon logica die op een blockchain wordt gepubliceerd, kan dergelijke transacties ontvangen of uitvoeren elk adres (transacties kunnen worden afgewezen of vereisen speciale argumenten om te functioneren) en dat kan fungeren als een onveranderlijke overeenkomst. Het doel van de slimme contracten is om op te treden als een "geautomatiseerd transactieprotocol dat de voorwaarden van een contract uitvoert" (Szabo, 1994) en werd voor het eerst bedacht door cryptograaf Nick Szabo. Het basisidee, en de bron van het contractdeel in de naam, is dat bepaalde delen van contracten kunnen zodanig in de software worden opgenomen dat de inbreuk daarop ofwel duur is of onmogelijk. Slimme contracten worden vaak verward met Ricardiaanse contracten (Griggs, 2015), de digitale opname en verbinding met andere systemen van een contract op wet. Dit is niet wat met slimme contracten wordt bedoeld, omdat ze niet legaal hoeven te zijn op geen enkele manier, noch verbonden met externe systemen. Men zou zich echter waarde kunnen voorstellen in de koppeling van slimme contracten met Ricardiaanse om de functionaliteit van "uit te besteden" juridische contracten met slimme contracten
%
%Basically, a smart contract is a computer application that can auto- matically execute commercial transactions and agreements. It also enforces the obligations of all parties in a contract without the added expense of an intermediary [14]. A smart contract also provides a means for owners of assets to pool their resources and create a cor- poration on the blockchain, where the articles of incorporation are coded into the contract, clearly spelling out and enforcing the rights of those owner. Associated agency employment contracts could define the decision rights of managers by coding what they could and could not do with corporate resources without ownership permission [79].

\chapter{De architectuur van het proof of concept}\label{chap: currentState}
Dit hoofdstuk bevat de resultaten van het onderzoek naar de geschikte software architectuur voor het proof of concept en beantwoord hierdoor de vraag \researchQuestionTwo (\textbf{SRQ2} uit paragraaf \ref{chap:researchQuestions}). Om hier antwoord op te geven, wordt de vraag in de volgende sub vragen verder opgesplitst in: 
\begin{enumerate}
	\item Wie zijn stakeholders van het proof of concept en wat zijn hun algemene belang?
	\item Wat zijn de requirements?
	\item Welke technologieën beste geschikt voor de proof of concept use case?
\end{enumerate}

De resultaten in dit hoofdstuk defineren het design van het proof of concept (PoC). De technologie keuzes worden in paragraaf \ref{decisionForcesView} behandeld. De C4 \footnote{https://c4model.com/} software architectuur modeleer methodiek, die tijdens de studie is gehanteerd wordt hierbij gebruikt.

\newpage

\section{Functionele en non-functionele requirements}\label{requirements}
De scope van de implementatie beperkt zich tot de volgende gebruikers: verzekeraar, verzekerde, makelaar en administrator. Om zo compleet mogelijk alle functionele requirements te noteren zijn de user stories in de onderstaande tabel \ref{fig:userstories} vanuit de verschillende gebruikers perspectief geschreven. Hiermee beantwoorden we gelijk de vraag wie de stakeholders zijn in het systeem en wat hun belangen zijn.\par

Er is tijdens het opstellen van deze requirements gekozen om alleen het essentiële op te schrijven en deze zoveel mogelijk de versimpelen. Te weten dat, het quality attributes zoals het gebruiksgemak en security eisen van de PoC op een rendabel niveau moeten zijn.

\begin{figure}[h!]
    \begin{center}
        \includegraphics[scale=0.7]{images/userstories}
        \caption{User stories die de functionele requirements defineren.}
        \label{fig:userstories}
    \end{center}
\end{figure}
\newpage

\subsection{Overige Requirements}
In de onderstaande opsomming staan overige requirements die ook aan het PoC worden gesteld.
\begin{itemize}
  \item \textbf{R1.} Het systeem geeft gebruikers toegang op basis van hun email en wachtwoord.
  \item \textbf{R2.} Het systeem controleert bij een claim aanvraag van of het polis nummer valide is
  \item \textbf{R3.} Het systeem moet automatisch goedkeuring geven voor een claim als het claimbedrag onder 1000 euro zit en het type diefstal is.
  \item \textbf{R4.} Met gebruik van smart contracts en de blockchain wordt de data integriteit gewaarborgd.
  \item \textbf{R5.} Een verzekeraar kan bij het aanmaken van een nieuwe polis aangeven wat de verdeling is tussen de verzekeringsmaatschappijen.
  \item \textbf{R6.} Het systeem ondersteund Ruitschade, Brandschade, Stormscade en diefstal als types voor een claim aanvraag.
  \item \textbf{R7.} Een claim is via de blockchain te verifiëren.
  \item \textbf{R8.} Nadat alle verzekeraars een claim aanvraag hebben goed gekeurd word de claim automatisch op status goedgekeurd gezet.
  \item \textbf{R9.} Een claim kan alleen de open, in behandeling, goedgekeurd, afgewezen en automatisch goedgekeurd statussen hebben.
  \item \textbf{R10.} Het systeem geeft per claim aan op hoeveel goedkeuringen het wacht en nog nodig heeft.
  \item \textbf{R11.} Wanneer een verzekeraar een claim weigert word de claim automatisch op status afgewezen gezet.
  \item \textbf{R12.} A transcation should not take more than 5 minutes.
  \item \textbf{R13.} Het systeem slaat de claims gedecentraliseerd op. Er zijn dus geen centrale punten van controle.
  \item \textbf{R14.} Het systeem schaalt met de groei van gebruikers mee zonder dat het systeem langzamer word.
  \item \textbf{R15.} Gebruikers hebben interactie met het systeem via een web interface.
\end{itemize}
\newpage

\subsection{Context - Actors en hun High-level Use Cases}
In het onderstaande diagram zijn de actors te zien. Het laat zien hoe deze gebruikers/systemen met het systeem communiceren.
\begin{figure}[h!]
    \begin{center}
        \includegraphics[width=\paperwidth-200]{images/context}
        \caption{C4 - Context.}
        \label{fig:c4Context}
    \end{center}
\end{figure}

In de onderstaande tabel worden de actors beschreven.
\begin{figure}[h!]
    \begin{center}
        \includegraphics[width=\paperwidth-250]{images/actors}
        \caption{Actors in het systeem}
        \label{fig:actors}
    \end{center}
\end{figure}

\newpage

\section{Decision forces view}\label{decisionForcesView}
Een decision force is een term die ik tijdens het Advanced Software Development semester heb geleerd. Een force is een beslissingsfactor van een architectureel probleem dat zich in het systeem of zijn omgeving. In de context van dit project is het de blockchain technologie keuze waarop het proof of concept op wordt gebouwd. Deze decision forces kunnen vanuit viewpoints komen, zoals: operationeel, ontwikkeling, zakelijk, organisatorisch, politiek, economisch, juridisch, regelgevend, ecologisch, sociaal \cite{architectureForces}. Uiteindelijk wordt een decision force gekoppeld aan een system quality attribute die de verschillende concerns aantoont. Hiermee worden uiteindelijk de verschillende technologieën vergeleken. De volgende decision forces zijn in samenwerking met Allianz, de stakeholder van het proof of concept samengesteld naast de requirements die in paragraaf \ref{requirements} zijn behandeld.

\begin{itemize}
  \item \textbf{F1. Ervaring} Al ervaring met technologie keuze?
  \item \textbf{F2. Smart contracts} of de technologie smart contracts ondersteund?
  \item \textbf{F3. Is het gratis?} Is de technologie gratis om te gebruiken? 
  \item \textbf{F4. Documentatie} Wat is de staat van de documentatie. Bij problemen, hoeveel support is er?
  \item \textbf{F5. OS onafhankelijk} Draait het zowel op Mac en Linux? 
  \item \textbf{F6. Backing?} Hoeveel word het gebruikt voor een vergelijkbaar doeleinde?
  \item \textbf{F7. API} Is er een blockchain API beschikbaar?
\end{itemize}

Zie de bijlagen voor de decision forces view \ref{decisionForcesView} waarin een aantal technologieën worden vergeleken. Gezamenlijk met de conclusie uit paragraaf \ref{conclusionBlockchain} is er voor Ethereum als database technologie. Dit komt vooral door de kwaliteit van de Ethereum Decentralized Applications documentatie en het feit dat Ethereum momenteel de enige gepubliceerde smart contract ondersteunde blockchain is.\par

Voor de backend technologie is er gekozen voor SailsJS en Javascript (NodeJS). Dit omdat ik hier al ervaring mee heb en omdat de Truffle en Web3 API voor Ethereum geschreven is in Javascript. Voor de website is er gekozen voor AngularJS 2, aangezien ik hier ook ervaring in heb en dit niet het geval is met React of een ander web framework. Ervaring speelt in dit project vooral een rol gezien de beperkte tijd en gegeven de nieuwe technologieën.

\newpage

\section{Container diagram}

\begin{figure}[h!]
    \begin{center}
        \includegraphics[width=\paperwidth-100]{images/containers}
        \caption{C4 - Containers.}
        \label{fig:c4Containers}
    \end{center}
\end{figure}

\chapter{Implementatie}\label{chap:q2}
In dit hoofdstuk wordt het design van het proof of concept (PoC) die gebruik maakt van smart contracts en blockchain technlogie behandeld. Eerst worden de verschillende actors (gebruikers) gedefineerd waarbij de user stories worden vastgesteld, om te voorzien van de functionele requirements van het PoC.
\par

- quality attributes\newline
- c4.
%specifications for the PoC. Further descriptions of the PoC such as quality attributes and how to set the system of smart contracts up with a blockchain are also given. Thereafter a schematic of the prototypical interactions on the blockchain is shown along with the interactions between the smart contracts. The suggested solution to the described problem uses the decentralised, trust-less and immutable properties of blockchain technology as well as permissioning in the smart contracts. To be noted is, however, that no security or privacy liabilities outside of the blockchain have been resolved with this implementation. Some of the larger off-chain issues are mentioned
%in the discussion.
\newpage

\section{User stories en requirements}
De scope van de implementatie beschreven beperkt zich tot de volgende gebruikers:
verzekeraar, verzekerde en makelaar. Om zo compleet mogelijk alle functionele requirements te noteren is in de onderstaande tabel \ref{fig:userstories} de user stories vanuit de verschillende gebruikers perspectief geschreven. De verschillende gebruikerstypes worden daarna meer in detail gedefinieerd. Er is tijdens het opstellen van de requirements gekozen om alleen het essentiële op te schrijven en deze zoveel mogelijk de versimpelen. Te weten dat, het gebruiksgemak en security eisen van de PoC op een rendabel niveau moeten zijn.

\begin{figure}[h!]
    \begin{center}
        \includegraphics[scale=0.7]{images/userstories}
        \caption{User stories die de functionele requirements defineren en het development van het proof of concept leiden.}
        \label{fig:userstories}
    \end{center}
\end{figure}

\newpage

\begin{figure}[h!]
    \begin{center}
        \includegraphics[width=\paperwidth-100]{images/context}
        \caption{C4 - Context.}
        \label{fig:c4Context}
    \end{center}
\end{figure}

Verzekernemers zijn consumenten of bedrijven die een verzekering heeft en met een polisnummer claims kan aanvragen en de status van kan ophalen. De manier waarop gebruikers via de backend interactie hebben met de smart contract in de Ethereum blockchain word getoond in figuur \ref{fig:c4Context}.

\begin{figure}[h!]
    \begin{center}
        \includegraphics[width=\paperwidth-100]{images/containers}
        \caption{C4 - Containers.}
        \label{fig:c4Containers}
    \end{center}
\end{figure}

\begin{figure}[h!]
    \begin{center}
        \includegraphics[width=\paperwidth-100]{images/components}
        \caption{C4 - Containers.}
        \label{fig:c4Containers}
    \end{center}
\end{figure}

In de onderstaande opsomming staan de non-functinoele requirements die ook aan het PoC worden gesteld.
\begin{itemize}
  \item \textbf{R1.} Het systeem geeft gebruikers toegang op basis van hun email en wachtwoord
  \item \textbf{R2.} Het systeem controleert bij een claim aanvraag van of het polis nummer valide is
  \item \textbf{R3.} Het systeem moet automatisch goedkeuring geven voor een claim als het claimbedrag onder 1000 euro zit en het type diefstal is.
  \item \textbf{R4.} Met gebruik van smart contracts en de blockchain wordt de data integriteit gewaarborgd.
  \item \textbf{R5.} Een verzekeraar kan bij het aanmaken van een nieuwe polis aangeven wat de verdeling is tussen de verzekeringsmaatschappijen.
\end{itemize}

The design of the final PoC was based on the requirements and user stories mentioned above.
\chapter{Conclusie}\label{chap:conclusion}
%De resultaten van het onderzoek kan buiten de scope van dit onderzoek toegepast worden voor andere usecases.

%%%%%%%%%%%%%%%%%%%%%%%%%%%%%%%%%%%%%%%%%%%%%%%%%%%%%%%%%%%%%%%%%%%%%%%%%%%%%%
%%
%% Bibliography:
%%
%\cleardoublepage
%\phantomsection
\addcontentsline{toc}{chapter}{Bibliografie}
\bibliography{thesis}

% %%%%%%%%%%%%%%%%%%%%%%%%%%%%%%%%%%%%%%%%%%%%%%%%%%%%%%%%%%%%%%%%%%%%%%%%%%%%%%%%
% %% Appendix:
% %%

\chapter{Bijlagen}
\appendix
\chapter{Decision Forces View}\label{appendix:decisionForcesView}
\begin{figure}[h!]
    \begin{center}
        \includegraphics[width=\paperwidth-100]{images/decisonForcesView}
        \caption{Decision Forces View.}
    \end{center}
\end{figure}

\newpage

\chapter{Smart contract: Claims}\label{appendix:contract}
\begin{lstlisting}[language=Solidity]
pragma solidity ^0.4.15;

contract Claims {

    // Claim struct
    struct Claim {
        bytes32 id;
        string policyId;
        ClaimTypes claimType;
        uint amount;
        State state;
        bool exists;
        bytes32[] insurers;
    }

    struct Vote {
        bool value;
        bool hasVoted;
        bool canVote;
    }

    enum State {New, Pending, Approved, Rejected, AutoApproved}
    enum ClaimVote {Approved, Rejected}
    enum ClaimTypes {NA, GlasDamage, FireDamage, StormDamage, Theft}

    // Who owns this contract
    address private authority;

    // Claim[] claims;

    mapping(bytes32 => Claim) claims;
    mapping(bytes32 => mapping(bytes32 => Vote)) claimVotes;

    // Map of claims for addresses
    mapping(bytes32 => uint) claimsIndexMapping;
    mapping(address => bytes32[]) addressClaimsMapping;

    event newClaim(bytes32 id, string policyId, ClaimTypes claimType, uint amount);
    event claimApproved(bytes32 id);
    event claimRejected(bytes32 id);
    event claimAutoApproved(bytes32 id);
    event logVote(bool value, bool hasVotes, bool canVote);

    // Construct me
    function Claims() public {
        authority = msg.sender;
    }

    function createClaim(bytes32 id, string policyId, ClaimTypes claimType, uint amount, bytes32[] insurers) public {
        require(!hasClaim(id));
        var claim = Claim(
            id,
            policyId,
            claimType,
            amount,
            State.New,
            true,
            insurers
        );

        for (uint i = 0; i < insurers.length; i ++) {
            claimVotes[id][insurers[i]] = Vote(false, false, true);
        }

        claims[id] = claim;
        addressClaimsMapping[msg.sender].push(id);

        newClaim(id, policyId, claimType, amount);
        if (shouldAutoApprove(claim)) {
            claim.state = State.AutoApproved;
            claims[id] = claim;

            claimAutoApproved(claim.id);
        }
    }

    function getClaim(bytes32 id) public constant returns (
        string policyId,
        ClaimTypes claimType,
        uint amount,
        uint state
    ) {
        require(hasClaim(id));
        var claim = claims[id];

        policyId = claim.policyId;
        claimType = claim.claimType;
        amount = claim.amount;
        state = uint(claim.state);
    }

    function vote(bytes32 claimId, bytes32 insurerId, bool vote) public {
        require(hasClaim(claimId));

        var claim = claims[claimId];
        claim.state = State.Pending;
        claims[claimId] = claim;
        
        var voteObject = claimVotes[claimId][insurerId];
        require(!voteObject.hasVoted);

        voteObject.value = vote;
        voteObject.hasVoted = true;
        claimVotes[claimId][insurerId] = voteObject;

        if (hasVoteFinished(claimId)) {
            var approved = true;
            for (uint i = 0; i < claim.insurers.length; i++) {
                voteObject = claimVotes[claimId][claim.insurers[i]];
                
                if (!voteObject.value) {
                    approved = false;
                    i = claim.insurers.length + 1;
                }
            }
            
            if (approved) {
                claim.state == State.Approved;
                claimApproved(claimId);
            } else {
                claim.state == State.Rejected;
                claimRejected(claimId);
            }
            
            claims[claimId] = claim;
        }
    }

    function setClaimState(bytes32 id, State state) internal {
        var claim = claims[id];

        claim.state = state;
        claims[id] = claim;
    }

    function shouldAutoApprove(Claim claim) internal constant returns (bool) {
        return (
            (claim.claimType == ClaimTypes.Theft && claim.amount < 1000)
        );
    }

    function hasClaim(bytes32 id) internal constant returns (bool) {
        Claim storage claim = claims[id];
        return claim.exists == true;
    }
    
    function hasVoteFinished(bytes32 claimId) internal constant returns (bool) {
        var claim = claims[claimId];
        uint votes = 0;
        for (uint i = 0; i < claim.insurers.length; i++) {
            var vote = claimVotes[claimId][claim.insurers[i]];
        
            if (vote.hasVoted) {
                votes++;
            }
        }

        return votes == claim.insurers.length;
    }
}
\end{lstlisting}


%%%%%%%%%%%%%%%%%%%%%%%%%%%%%%%%%%%%%%%%%%%%%%%%%%%%%%%%%%%%%%%%%%%%%%%%%%%%%%%%
%% Index:
%%
% \printthesisindex

\end{document}
