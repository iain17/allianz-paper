%%%%%
%%
%% Onderzoeksverslag
%%
%% Version: v0.1
%% Authors: Iain Munro
%% Date: 18/02/2018
%%%%%

%Install packages using:
%sudo tlmgr install

% Available documentclass options:
%
%   <all `report` document class options, e.g.: `a5paper`>
%   withindex   - enables the index. New index entries can be added through `\index{my entry}`
%   glossary    - enables the glossary.
%   techreport  - typesets the thesis in the technical report format.
%   firstyr     - formats the document as a first-year report.
%   times       - uses the `Times` font.
%   backrefs    - add back references in the Bibliography section
%
% For more info see `README.md`
\documentclass[firstyr,a4paper,oneside]{cam-thesis}%withindex
\usepackage[dutch]{babel}
% Citations using numbers
\usepackage[numbers]{natbib}

\newcommand{\thesisTitle}{Allianz - Automatiseren Claim Process}

\usepackage[utf8]{inputenc}

%APA norm
\usepackage{babel} 
\usepackage{apalike}
\bibliographystyle{apalike}

%tables new lines
\usepackage{makecell}

%Set the title spacing correctly
\usepackage{titlesec}
\titlespacing{\chapter}{0pt}{0pt}{0pt}
\titlespacing{\section}{0pt}{0pt}{0pt}
\titlespacing{\subsection}{0pt}{0pt}{0pt}

%Om de pagina margins etc te debuggen
%\usepackage{showframe}

\usepackage{pgfgantt}
\usepackage{rotating}
\usepackage[graphicx]{realboxes}

%%%%%%%%%%%%%%%%%%%%%%%%%%%%%%%%%%%%%%%%%%%%%%%%%%%%%%%%%%%%%%%%%%%%%%%%%%%%%%%%
%% Style (Changing the visual style of chapter headings and stuff.)
%%
\RequirePackage{titlesec}
% [Fixes issue #34 (see https://github.com/cambridge/thesis/issues/34). Solution from: http://tex.stackexchange.com/questions/299969/titlesec-loss-of-section-numbering-with-the-new-update-2016-03-15
\RequirePackage{etoolbox}
\makeatletter
\patchcmd{\ttlh@hang}{\parindent\z@}{\parindent\z@\leavevmode}{}{}
\patchcmd{\ttlh@hang}{\noindent}{}{}{}
\makeatother
% end of issue #34 fix]
\newcommand{\PreContentTitleFormat}{\titleformat{\chapter}[display]{\scshape\Large}
{\Large\filleft\MakeUppercase{\chaptertitlename} \Huge\thechapter}
{1ex}
{}
[\vspace{1ex}\titlerule]}
\newcommand{\ContentTitleFormat}{\titleformat{\chapter}[display]{\scshape\huge}
{\Large\filleft\MakeUppercase{\chaptertitlename} \Huge\thechapter}
{1ex}
{\titlerule\vspace{1ex}\filright}
[\vspace{1ex}\titlerule]}
\newcommand{\PostContentTitleFormat}{\PreContentTitleFormat}
\PreContentTitleFormat

%Om dubbelen legen pagina's weg te halen.
\let\cleardoublepage=\clearpage

%%%%%%%%%%%%%%%%%%%%%%%%%%%%%%%%%%%%%%%%%%%%%%%%%%%%%%%%%%%%%%%%%%%%%%%%%%%%%%%%
%% Thesis meta-information
%%

%% The title of the thesis:
\title{Automatiseren Claim Process}

%% The full name of the author (e.g.: James Smith):
\author{Calum Iain Munro}

%% College affiliation:
\college{Software Development, ICA, VT}

%% College shield [optional]:
\collegeshield{CollegeShields/ICA}

%% Submission date [optional]:
\submissiondate{July, 2018}

%% You can redefine the submission notice [optional]:
\submissionnotice{
HBO bachelorscriptie:\\
\textbf{Versie: 1 (Draft)}\\
\textbf{Datum: \today}\\
\\~\\
\textbf{Gegevens opdrachtgever:}\\
Bedrijf:			HeadForward B.V.\\
Contactpersonen:	Dani\"el Siahaya\\
\\~\\
\textbf{Gegevens opleiding:}\\
Opleiding: HBO bachelor Informatica\\
School: Hogeschool van Arnhem en Nijmegen\\
Begeleider:	Misja Nabben\\
Assessor: Rein Harle\\
\\~\\
\textbf{Gegevens opdrachtnemer:}\\
Teamlid: Calum Iain Munro (549288)\\
}

%% Declaration date:
\date{Febuari, 2018}

%% PDF meta-info:
\subjectline{Blockchaion and smart contracts}%Computer Science
\keywords{Onderzoeksverslag scriptie Calum Iain Munro HAN}

% %%%%%%%%%%%%%%%%%%%%%%%%%%%%%%%%%%%%%%%%%%%%%%%%%%%%%%%%%%%%%%%%%%%%%%%%%%%%%%%%
% %% Abstract:
% %%

 \abstract{My abstract ...}

% %%%%%%%%%%%%%%%%%%%%%%%%%%%%%%%%%%%%%%%%%%%%%%%%%%%%%%%%%%%%%%%%%%%%%%%%%%%%%%%%
% %% Acknowledgements:
% %%
 \acknowledgements{My acknowledgements ...}

%%%%%%%%%%%%%%%%%%%%%%%%%%%%%%%%%%%%%%%%%%%%%%%%%%%%%%%%%%%%%%%%%%%%%%%%%%%%%%%%
%% Glossary [optional]:
%%
% \newglossaryentry{HOL}{
%     name=HOL,
%     description={Higher-order logic}
% }

%%%%%%%%%%%%%%%%%%%%%%%%%%%%%%%%%%%%%%%%%%%%%%%%%%%%%%%%%%%%%%%%%%%%%%%%%%%%%%%%
%% Inhoudsopgave:
%%
\begin{document}
%%%%%%%%%%%%%%%%%%%%%%%%%%%%%%%%%%%%%%%%%%%%%%%%%%%%%%%%%%%%%%%%%%%%%%%%%%%%%%%%
%% Title page, abstract, declaration etc.:
%% -    the title page (is automatically omitted in the technical report mode).
\frontmatter{}

%Normale paragraven
\setlength{\parindent}{0em}
\setlength{\parskip}{1em}

\chapter{Versiebeheer}
\small
\begin{center}
 \begin{tabular}{|c c c c|} 
 \hline
 Datum & Versie & Door wie & Aanpassing \\ [0.5ex] 
 \hline
 12-03-2018 & v0  & Iain Munro & Eerste opzet \\
 \hline
\end{tabular}
\end{center}
\end{small}
\chapter{Voorwoord}
TODO.
\chapter{Samenvatting}
TODO.
\chapter{Inleiding}
Allereerst word in dit hoofdstuk het onderwerp van deze scriptie behandeld. Dit word gedaan door eerst in paragraaf \ref{chap:motivation} de aanleiding van het onderzoek te bespreken. Waarna de relevantie in  paragraaf \ref{chap:relevance} wordt besproken en aansluitend in
paragraaf \ref{chap:researchQuestions} de doel- en vraagstellingen zijn geformuleerd.

Het verdere verslag bestaat uit de resultaten van het onderzoek. Het begint met de eerste deelvraag waar de resultaten van het algemene onderzoek naar een aantal basis technische termen in blockchain-technologie en gerelateerde concepten zoals smart contracts word gedaan. Hierna worden er gekeken naar de implementatie van het proof of concept, door de requirements te onderzoeken zodat er in het laatste gedeelte naar de staat van de blockchain technologie gekeken kan worden en een aantal beslissingen naar de oplossingsrichting gemaakt kunnen worden.

\section{Aanleiding}\label{chap:motivation}
Zoals al aangegeven in het plan van aanpak verzekeren verzekeringsmaatschappijen zoals Allianz panden voor miljoenen. Dit type verzekeringen wordengedeeld met meerdere verzekeraars, om zo het risico te verspreiden. Dit principe heet co-insurance en het probleem hiermee en ook gelijk de aanleiding voor dit onderzoek is dat het claimproces te veel tijd kost voordat deze wordt uitgekeerd naar de klant. Waardoor klanten van Allianz ontevreden zijn. Dit komt omdat dit proces door de verschillende instanties op verschillende handmatige manier worden uitgevoerd. Het proces wordt bijvoorbeeld bij Allianz gedaan met Excel bestanden, maar dit verschilt per verzekeringmaatschappij.
Een claim kan dus vaak meer dan 3 maanden duren voordat deze werkelijk wordt uitbetaald.

\section{Relevantie}\label{chap:relevance}
De relevantie van dit onderzoek is om de laatste technologie op software gebied te onderzoeken om hiermee een proof of concept te ontwikkelen. In dit geval heeft de opdrachtgever aangegeven om in dit onderzoek naar de blockchain en smart contracts te willen kijken.

\newpage

\section{Probleemstelling}\label{chap:researchQuestions}
%Wat is de huidige situatie. Wat is de gewenste situatie? Wat is het verschil tussen de huidige en gewenste situatie?
Het doel van deze scriptie is om aan te tonen hoe blockchaintechnologie en smart contracts gebruikt kunnen worden om informatie over claims van verzekeringen veilig te delen en te controleren tussen partijen die elkaar niet noodzakelijk vertrouwen.\par
Dit wordt bewezen door een proof-of-concept software applicatie voor de use case van elektronische verzekeringgegevens. De resultaten van het onderzoek kan buiten de scope van dit onderzoek toegepast worden voor andere usecases.
\par
De hoofdvraag van dit onderzoek is (\textbf{MRQ}):\\
\textbf{MRQ - \researchQuestionMain} Deze vraag is onderverdeeld in verschillende deelvragen (\textbf{SRQ}):
\begin{itemize}
	\item \textbf{SRQ1: \researchQuestionOne} literatuuronderzoek naar een aantal basis technische termen in blockchain-technologie en gerelateerde concepten zoals smart contracts word gedaan.
  \item \textbf{SRQ2: \researchQuestionTwo} Om deze vraag te beantwoorden, zal ook een literatuuronderzoek worden uitgevoerd naar de verschillende implementaties van de blockchain technologie en beslissingen worden genomen op basis van eigenschappen die belangrijk zijn voor het ontwikkelen van de Proof of Concept.
  \item \textbf{SRQ3: \researchQuestionThree} Om deze vraag te beantwoorden word er samen met de klant Allianz gekeken naar het claim proces en worden er een software architectuur opgebouwd.
\end{itemize}

Het proof of concept (PoC) in dit verslag zal alleen bestaan uit de code die nodig is voor de smart contracts. De smart contracts maken het grootste deel uit van de core business logica en autorisatie. Aangezien de korte projectduur van 3 maanden en het overvloed aan bestaande blockchain en smart contract implementaties is er gekozen om geen blockchain te programmeren.
\newpage
\chapter{Wat is de blockchain}\label{chap:q1}
 Cool\par


%%%%%%%%%%%%%%%%%%%%%%%%%%%%%%%%%%%%%%%%%%%%%%%%%%%%%%%%%%%%%%%%%%%%%%%%%%%%%%
%%
%% Bibliography:
%%
%\cleardoublepage
%\phantomsection
\addcontentsline{toc}{chapter}{Bibliografie}
\bibliography{thesis}

% %%%%%%%%%%%%%%%%%%%%%%%%%%%%%%%%%%%%%%%%%%%%%%%%%%%%%%%%%%%%%%%%%%%%%%%%%%%%%%%%
% %% Appendix:
% %%

\chapter{Bijlagen}
\appendix
\chapter{Competenties}\label{chap:competenties}
\subsection{SD-1: Software Requirements}
De student analyseert en specificeert de eisen aan een ICT-oplossing op basis van de gebruikersbehoeften op een gestructureerde en gestandaardiseerde manier. Valideert de opgestelde eisen.
Beheert (veranderende) eisen tijdens het softwareontwikkeltraject.

\subsection{SD-2: Software Design}
De student kan op basis van de requirements de interne structuur – de elementen en hun relaties - van een data- intensief en gedistribueerd softwaresysteem bepalen, zowel op top-level niveau (architectuur) als ook op gedetailleerd niveau (ontwerp).
\\
De student kan de gemaakte ontwerpkeuzes onderbouwen, past tijdens het ontwerpen standaard notaties en best practices uit het beroepenveld toe, en houdt in het ontwerp rekening met mogelijke onderhoudsvragen.

\subsection{SD-3: Software Architecture}
De student kan op basis van de non-functional requirements de interne structuur op top-level niveau van een data-intensief en gedistribueerd softwaresysteem bepalen.
\\
De student kan de gemaakte architecturele keuzes onderbouwen en past tijdens het ontwerpen van de architectuur best practices uit het beroepenveld toe.

\subsection{SD-4: Software Construction}
De student kan op basis van een ontwerp werkende en betekenisvolle data- intensieve en gedistribueerde software systemen realiseren, schrijft begrijpbare en hoogwaardige source code en past professionele tools en technieken toe om dit te bereiken, en kan in teamverband een volledig geïntegreerd en systeem opleveren, dat klaar is voor ingebruikname

\subsection{SD-5: Software Testing and Quality}
De student kan aantonen dat het systeem aan de geïdentificeerde requirements voldoet en dat de opgeleverde producten, onder andere de source code, aan vooraf gedefinieerde kwaliteitscriteria voldoen.

\subsection{SD-6: Software Engineering Process and Management}
De student kan in een multidisciplinaire omgeving op grond van de gekozen ontwikkelmethodiek, passend bij de context en inhoud van de opdracht, een software-ontwikkeltraject projectmatig inrichten en uitvoeren, kiest geschikte methoden en technieken, past deze toe, en bewaakt de voortgang van het project door gebruik te maken van procesondersteunende tools.

\subsection{SD-7: Research}
De student kan een probleem op het terrein van Software Development (bijvoorbeeld inzet van nieuwe technologieën) oplossen door een kleinschalig onderzoek uit te voeren op een systematische, methodisch verantwoorde wijze, en kan de conclusies daaruit onderbouwen en effectief communiceren.

\subsection{SD-8: Self support}
De student kan als een beginnende professional zelfstandig een authentieke beroepsopdracht uitvoeren die leidt tot een of meer beroepsproducten en de uitvoering ervan verantwoorden.



%%%%%%%%%%%%%%%%%%%%%%%%%%%%%%%%%%%%%%%%%%%%%%%%%%%%%%%%%%%%%%%%%%%%%%%%%%%%%%%%
%% Index:
%%
% \printthesisindex

\end{document}
